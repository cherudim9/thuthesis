\chapter{新算法的实验结果}
\label{cha:exp}

\section{IBM的测试数据集}

这一小节讲介绍测试数据集中关于供电网络的模型假设。此数据集来源于IBM的公开数据~\cite{nassif2008power}。

此数据集共有六个电路参考设计,分别命名为IBMPG1到IBMPG6。表格\ref{tab:tabibm}总结了这几个设计的重要特征参数,包括电路的节点数、包含的纯电阻个数、包含的电压源与电流源个数、
金属层数等。整个数据集的节点数量之间的差距在两个数量级之间。

\begin{table}[htbp]
\centering
\caption{IBM供电网络测试集合}
\label{tab:tabibm}
\begin{tabular}{|c|r|r|r|r|r|r|}
\toprule[1.5pt]
\hline
数据编号 & 电流源数量 & 节点数量 & 电阻数量 & 短路数量 & 电压源数量 & 金属层数 \\
\hline
IBMPG1 & 10774 & 30638  & 30027  & 14208  &  14308 & 2\\
\hline
IBMPG2 & 37926 & 127238 & 208325 & 1298  & 330 & 5\\
\hline
IBMPG3 & 201054 & 851583 & 1401572 & 461 & 955 &  5\\
\hline
IBMPG4 & 276976 & 953583 & 1560645 & 11682 & 962  & 6\\
\hline
IBMPG5 & 540800 & 1079310 & 1076848 & 606587 & 539087 &  3\\
\hline
IBMPG6 & 761484 & 1670494 & 1649002 & 836107 & 936239 &  3\\
\hline
\end{tabular}
\end{table}

数据集文件用SPICE格式存储。具体的格式说明如下:

\begin{itemize}
\item 如果一行以星号(*)开头,那么这一行是注释;
\item 每一层电路的数据开头有一行注释:\emph{layer: <层名>,<网层> net: <网层序号>},例如\emph{layer: M1,Vdd net: 1}与\emph{layer: M1,Vss net: 2};
\item 层与层之间的连接(vias)数据前面会有一行注释:\emph{vias from: <网层序号> to <网层序号>};
\item 电路的节点形式为:\emph{n<网格序号>\_<节点x坐标>\_<节点y坐标>},例如"n3\_124\_921";
\item 如果一行的第一个字母是R,那么这一行的格式形如"Rxxxx Node1 Node2 value",其中“xxxx”是该电阻的名字(没有实际作用。可以忽略),表示Node1与Node2之间
有一个阻值为value的电阻;
\item 如果一行的第一个字母是V,那么这一行的格式形如"Vxxxx Node1 Node2 value",其中“xxxx”是该vias的名字(没有实际作用。可以忽略),表示Node1与Node2之间
有一个阻值为value的vias;value的值可以为0,表示这两个节点被短路了(通常用在两层之间相同位置的节点上);
\item 如果一行的第一个字母是r,那么这一行的格式形如"rxxxx Node1 Node2 value",其中“xxxx”是该电阻的名字(没有实际作用。可以忽略),表示Node1与Node2之间
有一个阻值为value的电阻;小写字母r开头的电阻与大写字母R的电阻的不同之处在于,小写字母r表示的是节点Node1和某个package(也就是Node2)(?)之间的电阻;例如Node1
如果名字是"n3\_123\_456",那么Node2一定是"\_X\_n3\_123\_456",所以这一行的形式为"rxxxx n3\_123\_456 \_X\_n3\_123\_456"。特殊的,电路数据文件里会指定这个package
的电压值,形式为"vxxxx Node2 0 value"其中Node2为小写字母r开头的数据中指定的package,value为该Node2的电压值;
\item 最后,负载电流源的形式为"iBxxxx Node 0 value"或者"iBxxxx 0 Node value“;分别表示从节点到Vdd的电流源大小或者从Vss到节点的电流大小。
\end{itemize}

\begin{figure}[H]
  \centering
  \includegraphics[height=8cm]{sample1}
  \caption{IBM供电网络测试数据集的数据样例}
  \label{fig:figsample1}
\end{figure}

图\ref{fig:figsample1}是一个比较小的供电网络例子。这个例子总共有两层金属网格,分别是水平排列的M1和垂直排列的M2;每一层金属网格都有4个Vdd线路和3个Vss线路;此供电网络
有两个package(用黑色实心圆圈表示),一个接电源,另一个接地。

它的数据如下:

\begin{lstlisting}
rr0 n3_0_0 _X_n3_0_0 0.5
v1 _X_n3_0_0 0 1
rr2 n2_125_125 _X_n2_125_125 0.5
v3 _X_n2_125_125 0 0
* layer: M1,VDD net: 1
R4 n1_0_0 n1_50_0 1.25
R5 n1_50_0 n1_100_0 1.25
R6 n1_100_0 n1_150_0 1.25
R7 n1_0_50 n1_50_50 1.25
R8 n1_50_50 n1_100_50 1.25
R9 n1_100_50 n1_150_50 1.25
R10 n1_0_100 n1_50_100 1.25
R11 n1_50_100 n1_100_100 1.25
R12 n1_100_100 n1_150_100 1.25
R13 n1_0_150 n1_50_150 1.25
R14 n1_50_150 n1_100_150 1.25
R15 n1_100_150 n1_150_150 1.25
* vias from: 1 to 3
V16 n1_0_0 n3_0_0 0.0
V17 n1_0_50 n3_0_50 0.0
V18 n1_0_100 n3_0_100 0.0
V19 n1_0_150 n3_0_150 0.0
V20 n1_50_0 n3_50_0 0.0
V21 n1_50_50 n3_50_50 0.0
V22 n1_50_100 n3_50_100 0.0
V23 n1_50_150 n3_50_150 0.0
V24 n1_100_0 n3_100_0 0.0
V25 n1_100_50 n3_100_50 0.0
V26 n1_100_100 n3_100_100 0.0
V27 n1_100_150 n3_100_150 0.0
V28 n1_150_0 n3_150_0 0.0
V29 n1_150_50 n3_150_50 0.0
V30 n1_150_100 n3_150_100 0.0
V31 n1_150_150 n3_150_150 0.0
* layer: M2,VDD net: 3
R32 n3_0_0 n3_0_50 1.25
R33 n3_0_50 n3_0_100 1.25
R34 n3_0_100 n3_0_150 1.25
R35 n3_50_0 n3_50_50 1.25
R36 n3_50_50 n3_50_100 1.25
R37 n3_50_100 n3_50_150 1.25
R38 n3_100_0 n3_100_50 1.25
R39 n3_100_50 n3_100_100 1.25
R40 n3_100_100 n3_100_150 1.25
R41 n3_150_0 n3_150_50 1.25
R42 n3_150_50 n3_150_100 1.25
R43 n3_150_100 n3_150_150 1.25
* layer: M1,GND net: 0
R44 n0_25_25 n0_75_25 1.25
R45 n0_75_25 n0_125_25 1.25
R46 n0_25_75 n0_75_75 1.25
R47 n0_75_75 n0_125_75 1.25
R48 n0_25_125 n0_75_125 1.25
R49 n0_75_125 n0_125_125 1.25
* layer: M2,GND net: 2
R50 n2_25_25 n2_25_75 1.25
R51 n2_25_75 n2_25_125 1.25
R52 n2_75_25 n2_75_75 1.25
R53 n2_75_75 n2_75_125 1.25
R54 n2_125_25 n2_125_75 1.25
R55 n2_125_75 n2_125_125 1.25
* vias from: 0 to 2
V56 n0_25_25 n2_25_25 0.0
V57 n0_25_75 n2_25_75 0.0
V58 n0_25_125 n2_25_125 0.0
V59 n0_75_25 n2_75_25 0.0
V60 n0_75_75 n2_75_75 0.0
V61 n0_75_125 n2_75_125 0.0
V62 n0_125_25 n2_125_25 0.0
V63 n0_125_75 n2_125_75 0.0
V64 n0_125_125 n2_125_125 0.0
*
iB0_0_v n1_0_0 0 0.3125m
iB0_0_g 0 n0_25_25 0.3125m
iB0_1_v n1_0_50 0 0.3125m
iB0_1_g 0 n0_25_25 0.3125m
iB0_2_v n1_0_100 0 0.3125m
iB0_2_g 0 n0_25_75 0.3125m
iB0_3_v n1_0_150 0 0.3125m
iB0_3_g 0 n0_25_125 0.3125m
iB0_4_v n1_50_0 0 0.3125m
iB0_4_g 0 n0_25_25 0.3125m
iB0_5_v n1_100_0 0 0.3125m
iB0_5_g 0 n0_75_25 0.3125m
iB0_6_v n1_50_50 0 0.3125m
iB0_6_g 0 n0_25_25 0.3125m
iB0_7_v n1_50_100 0 0.3125m
iB0_7_g 0 n0_25_75 0.3125m
iB0_8_v n1_100_50 0 0.3125m
iB0_8_g 0 n0_75_25 0.3125m
iB0_9_v n1_100_100 0 0.3125m
iB0_9_g 0 n0_75_75 0.3125m
iB0_10_v n1_50_150 0 0.3125m
iB0_10_g 0 n0_25_125 0.3125m
iB0_11_v n1_100_150 0 0.3125m
iB0_11_g 0 n0_75_125 0.3125m
iB0_12_v n1_150_0 0 0.3125m
iB0_12_g 0 n0_125_25 0.3125m
iB0_13_v n1_150_50 0 0.3125m
iB0_13_g 0 n0_125_25 0.3125m
iB0_14_v n1_150_100 0 0.3125m
iB0_14_g 0 n0_125_75 0.3125m
iB0_15_v n1_150_150 0 0.3125m
iB0_15_g 0 n0_125_125 0.3125m
\end{lstlisting}

\section{实验平台}

实验使用的集群由8个节点组成,每个节点是一个Xeon E5-2699v4。Xeon E5-2699v4属于Intel Xeon®系列的处理器,有22个核心,基础频率是
2.20GHz,最高可超频至3.60GHz;并有55MB的智慧缓存(Smart Cache),支持DDR4 SDRAM内存。节点之间用Enhanced Data Rate(EDR)的Infiniband连接着,
EDR的传输速率可以达到100Gb/s。每一个节点有128GB的内存。

\section{预条件子算法与并行化的实验结果分析}

本文使用C++语言实现第三章中的算法,编译器是Intel CC 17.0.4,MPI库使用的是Intel MPI 2017.3.196。

程序时间的测量使用的是系统时钟,运行时间没有包括数据读入和输出的时间。在正确性校验中,比较的标准是计算出来的电压值不能与标准值(由IBM测试集提供)差距超过$10^{-3}$,
也就是1毫伏。

对于每个测试点,我们分别用1个、2个、4个、8个节点对其进行多进程测试,以此来考察并行化对算法的加速效果。

\begin{sidewaystable}[htbp]
\centering
\caption{IBM数据集的测试结果}
\label{tab:tabibmresult}
\begin{tabular}{c|r|r|r|r|r|r|r|r}
\toprule[1.5pt]
\hline
数据编号 & 节点数量 & 金属层数 & 迭代次数 & 运算节点 & 通讯总量 & 非零元数量 & 通讯总量/非零元数量 & 运算时间(秒)  \\
\hline
IBMPG1 & 16327 & 4  & 172  & 1  &  0 & 75827 & 0\% & 0.427\\
\hline
& & & 238 & 2 & 3438 & 75827 & 4.53\% & 0.119 \\
\hline
& & & 254 & 4 & 6810 & 75827 &  8.98\% &  0.068 \\
\hline
& & & 261 & 8 & 11208 & 75827 & 14.78\% & 0.081 \\
\hline
IBMPG2 & 126905 & 8 & 244 & 1 & 0 & 542895 & 0\% & 3.003 \\
\hline
& & & 394 & 2 & 163410 & 542895 & 30.01\% & 1.291 \\
\hline
& & & 405 & 4 & 172544 & 542895 & 31.78\% &  0.700 \\
\hline
& & & 488 & 8 & 185346 & 542895 & 34.13\% & 0.459 \\
\hline
IBMPG3 & 285344 & 15 & 271 & 1 & 0 & 1508802 & 0\% & 9.663 \\
\hline
& & & 412 & 2 & 18660 & 1508802 & 1.23\% & 5.241 \\
\hline
& & & 471 & 4 & 262662 & 1508802 & 17.41\% & 3.600 \\
\hline
& & & 510 & 8 & 319344 & 1508802 & 21.17\% & 2.447 \\
\hline
IBMPG4 & 952618 & 12 & 337 & 1 & 0 & 4054056 & 0\% & 41.532 \\
\hline
& & & 563 & 2 & 1273824 & 4054056 & 31.421\% & 29.840 \\
\hline
& & & 576 & 4 & 2136810 & 4054056 & 52.71\% & 19.695 \\
\hline
& & & 568 & 8 & 2250576 & 4054056 & 55.51\% & 12.198 \\
\hline
IBMPG5 & 540220 & 15 & 336 & 1 & 0 & 2691958 & 0\% & 25.119 \\
\hline
& & & 466 & 2 & 116832 & 2691958 & 4.34\% & 13.088 \\
\hline
& & & 528 & 4 & 155100 & 2691958 & 5.76\% & 8.757 \\
\hline
& & & 523 & 8 & 177632 & 2691958 & 6.60\% & 5.459 \\
\hline
IBMPG6 & 834252 & 6 & 390 & 1 & 0 & 4125504 & 0\% & 43.936 \\
\hline
& & & 582 & 2 & 11148 & 4125504 & 0.27\% & 26.389 \\
\hline
& & & 571 & 4 & 227452 & 4125504 & 5.51\% & 15.453 \\
\hline
& & & 558 & 8 & 339142 & 4125504 & 8.22\% & 9.834 \\
\hline
\end{tabular}
\end{sidewaystable}

\section{总结}
