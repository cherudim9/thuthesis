\thusetup{
  %******************************
  % 注意:
  %   1. 配置里面不要出现空行
  %   2. 不需要的配置信息可以删除
  %******************************
  %
  %=====
  % 秘级
  %=====
  secretlevel={秘密},
  secretyear={10},
  %
  %=========
  % 中文信息
  %=========
  ctitle={大规模供电网络分析的并行加速算法},
  cdegree={},
  cdepartment={计算机科学与技术系},
  cmajor={计算机科学与技术},
  cauthor={龙浩民},
  csupervisor={蔡懿慈教授},
  % 日期自动使用当前时间,若需指定按如下方式修改:
  % cdate={超新星纪元},
  %
  % 博士后专有部分
  cfirstdiscipline={计算机科学与技术},
  cseconddiscipline={系统结构},
  postdoctordate={2009年7月——2011年7月},
  id={编号}, % 可以留空: id={},
  udc={UDC}, % 可以留空
  catalognumber={分类号}, % 可以留空
  %
  %=========
  % 英文信息
  %=========
  etitle={An Introduction to \LaTeX{} Thesis Template of Tsinghua University v\version},
  % 这块比较复杂,需要分情况讨论:
  % 1. 学术型硕士
  %    edegree:必须为Master of Arts或Master of Science(注意大小写)
  %             “哲学、文学、历史学、法学、教育学、艺术学门类,公共管理学科
  %              填写Master of Arts,其它填写Master of Science”
  %    emajor:“获得一级学科授权的学科填写一级学科名称,其它填写二级学科名称”
  % 2. 专业型硕士
  %    edegree:“填写专业学位英文名称全称”
  %    emajor:“工程硕士填写工程领域,其它专业学位不填写此项”
  % 3. 学术型博士
  %    edegree:Doctor of Philosophy(注意大小写)
  %    emajor:“获得一级学科授权的学科填写一级学科名称,其它填写二级学科名称”
  % 4. 专业型博士
  %    edegree:“填写专业学位英文名称全称”
  %    emajor:不填写此项
  edegree={Doctor of Engineering},
  emajor={Computer Science and Technology},
  eauthor={Xue Ruini},
  esupervisor={Professor Zheng Weimin},
  eassosupervisor={Chen Wenguang},
  % 日期自动生成,若需指定按如下方式修改:
  % edate={December, 2005}
  %
  % 关键词用“英文逗号”分割
  ckeywords={VLSI,供电网络,线性方程组,并行化算法,多重网格},
  ekeywords={VLSI, Power Grid, Linear System, Parallel Algorithm, Multigrid Preconditioner}
}

% 定义中英文摘要和关键字
\begin{cabstract}


  随着大规模集成电路的飞速发展,其供电网络的规模与日俱增,它的计算成本也越来越高。供电网络的仿真设计作为电路设计中最重要的一环之一,影响着电路设计的成败。
  因此供电网络的仿真设计对计算机的计算效率有着非常高的要求。而供电网络的仿真本质上是一个线性方程组的求解问题,解决的方法主要有直接求解法和迭代求解法,都依赖于
  大规模的矩阵、向量运算。

  本文针对供电网络仿真数据的特点,运用了并行计算的技术,对主流的共轭梯度迭代法进行了改进,包括以下创新点:(1)把稀疏矩阵和向量的内容分发到
  计算集群里的不同节点,从而提高了矩阵与向量运算的并行度;(2)针对系数矩阵规模大的特点,选用了多重网格预条件子,使得预条件的计算效率更高。
  我们最后用C++语言与开源的并行计算库在多机集群上实现了一个高效的供电网络仿真算法,并进行了多项测试,证明了算法的有效性。

\end{cabstract}


\begin{eabstract}
   With the fast development of Very Large Scale Integrated circuit, the size of power grid of circuit has increased significantly and become
   more and more computationally challenging both in runtime and memory consumption. The design and simulation of power grid is vital to
   that of the integrated circuit since the former is the most basic and important step of the latter. So the design of power grid is in
   enormouse demand for computational efficiency and ability of the computers. While to solve this problem is to solve a system of linear
   equations, there are two ways to work it out, one is direct method and the other one is iterative method, both of which rely heavily on
   massive matrix and vector operations.

   In this paper, we refine the mainstream iterative method, Conjugate Gradient method by adopting parallel computing techniques and taking
   advantages of the mathematical features of the data in power grid. Our contributions are: (1) Dispatching consecutive contents in matrices
   and vectors to different nodes in a computing cluster, so that we boost the efficiency of matrices and vectors operations by parallel
   computing, (2) Adopting Multigrid preconditioner, which both largely improves the convergence of the linear system and requires relatively
   small amount of computation efforts, based on the fact that the matrix has very sparse coefficients. We implemented an effective algorithm
   with C++ language and with open source parallel computing library on a multi-node cluster. Our experiments on IBM benchmark proved our
   insight in the paper.
\end{eabstract}
