\chapter{相关工作与研究现状}
\label{cha:relatedworks}

\section{供电网络的分析与仿真}

大规模供电网络的仿真是一个很热门的问题。主流的算法可以分成两类:直接求解法和迭代求解法。直接求解法首先通过对矩阵进行分解,然后进行若干次回带便得到解向量;典型的方法有
KLU~\cite{davis2010algorithm}以及Cholmod~\cite{davis2005cholmod}等。
直接求解法通常更加鲁棒,而且求出来的逆矩阵还可以复用;但所需要的内存过多,随着问题规模的扩大而急剧增长,不能处理规模越来越大的供电网络。
迭代求解法通常先设定一个初始解,然后构造具有一定性质的迭代序列,如果问题满足一定的条件的话最终会收敛到方程的解。迭代求解法所需要的计算资源更少,尤其是遇到大规模的电路矩阵时;
然而迭代求解法的计算不够稳定,对于有病态性的问题收敛速度非常慢。2001年,Chen等人提出用预条件子的Krylov子空间迭代法~\cite{chen2001efficient},
这个方法比当时传统的直接求解方法要快了10倍左右,并且有更好的收敛性质。之后,各种预条件子都被提出来,包括随机行走~\cite{qian2005power},
多重网格方法~\cite{kozhaya2002multigrid},层次方法~\cite{zhao2002hierarchical}等。

为了促进大规模供电网络仿真分析的研究,IBM奥斯汀研究实验室在2011年举办了供电网络分析竞赛。竞赛中获得头名的是清华大学杨建磊等提出的PowerRush系统~\cite{yang2012powerrush},他们
使用的是基于代数多重网格的预条件共轭梯度求解算法。

\section{线性方程组的求解}

在高性能计算的领域,线性方程组的并行加速是一个被广泛研究但仍然十分活跃的问题。Saad的著作~\cite{saad2003iterative}中有对这个问题涉及到的并行数值计算以及各类
迭代算法有入门的介绍,里面介绍了:稀疏矩阵的矩阵向量乘法操作(Sparse Matrix-Vector Multiplication,SPVM),基于Jacobo或者SSOR预条件子的Krylov子空间方法等内容。
Lee等人的论文~\cite{lee2004performance}深入介绍了计算密集的SPVM操作的各种研究和优化。作者在论文里提出了一种对算法参数进行调节的方法,这个方法基于经验模型和一定程度的
搜索算法。

共轭梯度算法~\cite{hestenes1952methods}(Conjugate Gradient Method,CG)由Hestenes等人提出,在数值计算领域有着非常悠久而成功的应用历史。
Van der Sluis等人研究了CG算法的收敛性能~\cite{van1986rate}及其Ritz值的关系。

在描述网格型的拓扑结果,或者计算偏微分方程的离散化形式的时候,经常会出现规模巨大、系数十分稀疏的矩阵,因此SPVM操作在该类数值计算中十分重要。而当矩阵与向量的数据分布在
不同计算节点的时候,SPMV操作会带来巨大的通讯开销。所以在对数据进行节点分发的的时候,通常会选择一种分发方式,使得跨节点的数据访问尽可能地少出现。
Demmel等人的报告~\cite{demmel1993parallel}中提出了一些能够同时进行数据传输与数值计算(也就是两种操作并行进行)的预条件共轭梯度算法(Preconditioned Conjugate Gradient Method, PCG),从而在一定程度上掩盖了数据通讯时间降低
算法效率的问题。

近年来,越来越多研究开始尝试使用图形处理核心(Graphics Processing Unit,GPU)来解决非图形相关的问题,加速计算密集的任务。Owens等人的报告~\cite{owens2007survey}
介绍了这些趋势。
