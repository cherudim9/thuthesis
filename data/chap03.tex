\chapter{并行化加速的预条件共轭梯度算法}
\label{cha:algo}

\section{供电网络的抽象数学模型}

由于本数据集是瞬态的直流DC网络,所以网络中只有三种基本元件:电阻、独立电压源、独立电流源。为了建立改进节点分析方法的方程,先根据
电路里的连接关系建立系数矩阵$A$:对于连接节点$i$与节点$j$的支路$k$,如果这个支路的方向是从$i$到$j$,那么$A_{k,i}=1, A_{k,j}=-1$;
如果支路的方向是从$j$到$i$,那么$A_{k,i}=-1, A_{k,j}=1$;其余情况$A$的元素为0。

更进一步的,根据支路$b$是纯电阻支路、电压源支路、电流源支路可以把$A$按列分成三部分$A_G,A_V,A_I$。同样的对于支路电流向量$i$,可以分成$i_G,i_V,i_I$;
对于节点电压向量$v_b$,可以分成$v_G,v_V,v_I$。
\begin{align}
A=\begin{bmatrix} A_I \\ A_V \\ A_G \end{bmatrix}, \quad
i_b=\begin{bmatrix} i_I \\ i_V \\ i_G \end{bmatrix},\quad
v_b=\begin{bmatrix} v_I \\ v_V \\ v_G \end{bmatrix}
\end{align}
那么根据基尔霍夫电压定律和基尔霍夫电流定律可以写出($v_n$是节点电压向量):
\begin{align}
A^T i_b & =  0 \\
A v_n & = v_b
\end{align}
展开上述两个式子可得:
\begin{align*}
A_I^T i_I + A_V^T i_V + A_G^T i_G & = 0 \\
A_I v_n & = v_I \\
A_V v_n & = v_V \\
A_G v_G & = v_G \\
i_I & = I \\
v_V & = V \\
i_G & = P v_G \\
\end{align*}
其中$P$是电导邻接矩阵:如果支路$k$上存在阻值为$R$、连接$i$到$j$的电阻,那么$P_{k,i}=\frac{1}{R}$,$P_{k,j}=\frac{1}{R}$。

根据改进节点分析方法的原理,可以通过简化去掉所有支路电压的变量,也就是$v_b$;以及大部分支路电流变量,也就是$i_b$。
化简后可以得到:
\begin{align}
    A_G^T P A_G v_n + A_V^T i_V & = -A_I^T I    \\
    A_V v_n & =V
\end{align}
令$G=A_G^T P A_G$,其中$G$也就是在公式\ref{Geq}中提到的电导矩阵。那么可以得到:
\begin{align}
    \begin{bmatrix}
    G & A_V^T \\
    A_V & 0 \\
    \end{bmatrix}
    \begin{bmatrix}
    v_n \\ i_V
    \end{bmatrix}
    =
    \begin{bmatrix}
    0 & -A_I^T  \\
    1 & 0
    \end{bmatrix}
    \begin{bmatrix}
    V \\ I
    \end{bmatrix}
    \label{eqbasic}
\end{align}

式\ref{eqbasic}的右边是确定的,左边有一个未知向量$x=\begin{bmatrix}v_n \\ i_V\end{bmatrix}$,因此可以使用线性方程求解器进行求解。此外,系数矩阵
可以看出来非常稀疏并且是对角占优的,因此有很多优化的余地。但是,值得注意的是系数矩阵并不是对称正定的,所以像Cholesky Solver这种算法就不能再用了。

\subsection{处理节点间的短路}

由于系数矩阵使用的是电导值,所以节点间的短路并不能直接用阻值为0的电阻代替。为了解决这个问题,考虑对于一个电路来说,由短路的节点组成的集合等效于一个超节点,也就是说
我们可以把它们当做一个节点来处理。实现的时候,对于短路的节点集合,我们新建一个超节点表示它们的等效节点,与这个超节点相连的所有支路是这些短路的节点的支路的并集,这个
超节点挂载的电流源负载是这些短路的节点的电流源负载之和。

此外,为了求出超节点,在处理读入数据的时候,我们采用了并查集~\cite{tarjan1975efficiency}的方法,即维护一个图的森林,一开始所有节点都是森林里的孤立的节点;如果读入两个节点$i$和$j$是短路的,那么找到节点$i$所在的连通块的根节点,把这个根节点挂在节点$j$下,也就是成为$j$的子节点;最后所有节点的关系形成了一个森林,森林里每个连通块都是
一个超节点。

\section{解线性方程的共轭梯度算法}

求解式\ref{eqbasic}中的线性方程组有两种基本方法:直接求解法与间接迭代法。直接求解法的算法鲁棒性更强,但是由于该线性方程组的节点规模非常巨大,直接求解法会要求
过多的CPU计算资源与内存,无法解决问题。迭代法由于预条件子的问题,求解稳定性不如直接求解法,但是更加节省计算资源,是解决该问题的合适方法。对于本问题,我们采取经典的基于预条件子的共轭梯度法(Preconditioned Conjugate Gradient Method,PCG)进行求解。下文探讨的主要是对此算法的优化。

没有用预条件子的共轭梯度算法如下所示:
\begin{lstlisting}
//算法1.cpp
r[0] = b - A * x0;
p[0] = r[0];
k = 0;
repeat
    alpha[k] = dot(r[k], r[k]) / dot(p[k], A * p[k]);
    x[k+1] = x[k] + alpha[k] * p[k];
    r[k+1] = r[k] - alpha[k] * A * p[k];
    if r[k+1]足够小
        break;
    else
        beta[k] = dot(r[k+1], r[k+1]) / dot(r[k], r[k]);
        p[k+1] = r[k+1] + beta[k] * p[k];
        k = k + 1;
    end if
end until
求解结果为x[k+1]
\end{lstlisting}

但是共轭梯度算法的求解速度,也就是收敛速度取决于系数矩阵$A$的条件数。为了改善这一点,我们可以用一个预条件子$M$对系数矩阵$A$进行预处理,使得$M^{-1}A$的条件数尽量小。
使用预条件子后的算法如下:
\begin{lstlisting}
//算法2.cpp
r[0] = b - A * x0;
z[0] = inverse(M) * r[0]
p[0] = z[0];
k = 0;
repeat
    alpha[k] = dot(r[k], z[k]) / dot(p[k], A * p[k]);
    x[k+1] = x[k] + alpha[k] * p[k];
    r[k+1] = r[k] - alpha[k] * A * p[k];
    if r[k+1]足够小
        break;
    else
        z[k+1] = inverse(M) * r[k+1];
        beta[k] = dot(z[k+1], z[k+1]) / dot(z[k], z[k]);
        p[k+1] = z[k+1] + beta[k] * p[k];
        k = k + 1;
    end if
end until
求解结果为x[k+1]
\end{lstlisting}

\section{基于多重网格的预条件子}

\section{多机并行优化中的线程级优化}

实现算法的时候,我们在两个方面进行了并行优化:线程级和进程级的优化。实验时使用具有八个核心(CPU Processor)的机器,我们在每个核心上使用八个线程并行进行各种矩阵运算,包括
向量内积、矩阵与向量的乘法、向量加法等,从而最大程度地发掘算法的并行性。

首先,为了进行线程级的优化,我们直接使用了OpenMP代码库提供的API。OpenMP(Open Multi-Processing)是一套支持跨平台共享内存方式的多线程并发的编程API,使用C/C++和Fortran语言,可以在大多数的处理器体系和操作系统中运行具体的实现。OpenMP提供了对并行算法的高层的抽象描述,程序员通过在源代码中加入专用的pragma来指明自己的意图,由此编译器可以自动将程序进行并行化,并在必要之处加入同步互斥以及通信。当选择忽略这些pragma,或者编译器不支持OpenMP时,程序又可退化为通常的程序(一般为串行),程序码仍然可以正常运作,只是不能利用多线程来加速程序执行。一个典型的例子如下:
\begin{lstlisting}
//dot_product.cpp
int ComputeDotProduct(const int n, const Vector & x, const Vector & y, double & result) {
  result = 0.0;
  double * xv = x.values;
  double * yv = y.values;
  #pragma omp parallel for reduction (+:result)
  for (local_int_t i=0; i<n; i++)
    result += xv[i]*yv[i];
  return 0;
}
\end{lstlisting}

上述例子中,通过pragma指令,我们实现了向量内积算法中的线程级并行化。实现的方法是在可以并行执行的循环代码前加pragma指令,这样编译器就会自动生成相应的代码,在循环代码前
派生出若干个线程,不同线程之间并行执行,执行完后汇集到父线程上。
需要注意的是,由于不同线程的计算累加结果是分开存储的,所以要在pragma指令中指明对result变量进行reduction操作,从而把所有线程计算的结果累加存储在result变量中。
使用OpenMP之前,要注意判断代码中是否有数据依赖的部分;如果有,要尽量去掉,保证代码的可并行性。

\section{多机并行优化中的进程级优化}

虽然上一小节中提到的线程级的优化对程序效率已经有了不少提升,但是还远远不够。因为线程级的并行化要求共享内存,也就是所有数据都存储在本机上;此外,单机的
处理能力是有限的,本实验的数据规模较大,需要更多的计算资源。为此,需要进行多机的并行化处理,也就是进程级的并行优化。进程级的并行加速能最大程度地同时发掘多台机器的
计算能力,非常适合解决这种大规模的计算问题。但是与线程级的优化不同,进程与进程之间的内存是不共享的,因此进程与进程之间需要相互通信,从而同步计算的数据。对于进程级
的并行化来说,通信带来的时间与空间开销往往限制了算法的优劣程度以及算法的并行化程度。通常来说,进程、机器越多,通讯的花销越大,性能有可能不如使用进程更少的算法。
\begin{figure}[H]
  \centering
  \includegraphics[height=8cm]{paral_demon}
  \caption{MPI中的负载平衡}
  \label{fig:figparaldemon}
\end{figure}

对于本实验来说,我们需要把进程的并行化运用到共轭梯度算法中。注意到系数矩阵非常稀疏,并且具有局部性。
那么一种可行的办法是把矩阵中连续的行分配给每一个进程,如图\ref{fig:figparaldemon}所示:若系数矩阵的
大小为$N\times N$,共有$K$个进程,那么第一个进程计算$x_1,\ldots,x_{\frac{N}{k}}$,第二个进程计算计算$x_{\frac{N}{k}+1},\ldots,x_{2\frac{N}{k}}$,
依次类推。也就是对于进程$i$来说,他只负责计算和保存$x_{(i-1)\frac{N}{k}+1},\ldots,x_{i\frac{N}{k}}$的值。所以在第$i$个进程迭代更新$x$的值的时候,需要
通过通讯手段来获得其余$x$(不在这个进程上)的值。在程序中,由于这一点,我们不能再用变量x来访问整个向量,因为内存中只存有该向量的部分内容(下文中用
local\_x来表示本地进程中的x的值)。为了正确地执行原算法,我们必须在进程中进行必要的通讯以交换不同进程的内容。

对于共轭梯度算法来说,主要的数学运算与对应的通讯方法如下:

\begin{itemize}
\item 计算向量$x$和$y$的内积。由于本进程中只有local\_x与local\_y的内容,所以每个进程先计算各自local\_x与local\_y的内积,存为local\_result,即本地结果;然后
把各自的本地结果广播给其他进程,从而完成本地结果的累加,计算出全局的内积;
\item 计算向量$x$和$y$的和。直接计算即可,不需要通讯,因为向量加法对于各个元素来说是独立的;
\item 计算矩阵$A$和向量$x$的乘积$y$。因为计算结果$y$是局部的,也就是只需要计算$y_{(i-1)\frac{N}{k}+1},\ldots,y_{i\frac{N}{k}}$,所以只需存储系数矩阵A的第
$((i-1)\frac{N}{k}+1)$到$(i\frac{N}{k})$行;此外,如果存在非零系数$A_{i,j}$,那么计算结果时就需要$x_j$的值;所以我们需要预处理出每一个$y_i$所需要的非局部的
$x$值,并通过通讯手段从其他进程中获取其值。
\end{itemize}

改动后的算法3如下:
\begin{lstlisting}
//算法3.cpp
local_r[0] = local_b - local_A * local_x0;
syncGlobal(sync_r, local_r);
local_z[0] = inverse(M) * sync_r[0];
local_p[0] = local_z[0];
k = 0;
repeat
    //get global dot(r,z)
    local_rtz = dot(local_r[k], local_z[k]);
    getGlobal(global_rtz, local_rtz);
    //get global dot(p,Ap)
    syncGlobal(sync_p, local_p);
    local_Ap = local_A * sync_p;
    local_pAp = dot(local_p[k], local_Ap);
    getGlobal(global_pAp, local_pAp);
    //compute alpha
    alpha[k] = global_rtz / global_pAp;
    //compute x
    local_x[k+1] = local_x[k] + alpha[k] * local_p[k];
    local_r[k+1] = local_r[k] - alpha[k] * local_Ap[k];
    if r[k+1]足够小
        break;
    else
        syncGlobal(sync_r, local_r)
        local_z[k+1] = inverse(M) * sync_r[k+1];
        beta[k] = global_rtz / old_rtz;
        old_rtz = global_rtz;
        local_p[k+1] = local_z[k+1] + beta[k] * local_p[k];
        k = k + 1;
    end if
end until
求解结果为x[k+1]
\end{lstlisting}
其中,主要的通讯函数有:
\begin{enumerate}
    \item getGlobal(global\_data, local\_data)函数,作用是通过通讯手段从其他进程中获取、接受它们各自的local\_data值,并统计到global\_data变量中(合并结果的
    方法有累加、取最大值、取最小值等);
    \item syncGlobal(result, local\_data)函数,作用是通过通讯手段,从其他进程中获取、接受local\_data对应的变量的部分全局值,并存在result变量中;
\end{enumerate}

与单机上的原算法相比,使用了多进程并行化优化的算法3在矩阵运算方面处理的数据规模更小,比如所处理的向量长度从原来的$N$降到了$N/\text{处理器数量}$,因此减少了程序的
数学运算量;并且,利用矩阵与向量运算的各元素独立性,不同的进程之间可以并行执行各种运算,因此大大提高了程序运行的时间利用率;但是算法引入了另外一种开销:通讯开销。比如
在计算向量内积时,需要等待所有进程都计算完本地局部内积后再相加,才可以得出总体的内积;在计算矩阵与向量的乘积时,需要从其它进程中接收某些非本地向量元素,也需要把本地的
向量元素发送到其它有需要的进程里。可以看出,进程数越多,算法的并行度越高,计算本地结果的效率越高,但是花在通讯上的花销会更大;进程数越少,算法并行度越低,但是通讯代价
会降低。

在本实验中,多机的并行通讯采用了消息传递接口(Message Passing Interface,MPI)的标准。MPI是一个跨语言的通讯协议,用于编写并行计算机。支
持点对点和广播。MPI是一个信息传递应用程序接口,包括协议和和语义说明,他们指明其如何在各种实现中发挥其特性。

\section{本章小结}

本章介绍了供电网络的抽象数学模型的建立,以及解决其相应的线性方程组的方法。基于现有的共轭梯度算法,作者实现了多机多核并行的算法,并进一步分析了这个算法的性能。最后,为了提高算法的性能,作者又引入了基于多重网格的预条件子,从而解决了问题病态下的收敛速度的问题。
